\makeatletter
\def\e@alloc#1#2#3#4#5#6{%
  \global\advance#3\@ne
  \e@ch@ck{#3}{#4}{#5}#1%
  \allocationnumber#3\relax
  \global#2#6\allocationnumber
  }
\def\@pr@videpackage[#1]{%
  \expandafter\xdef\csname ver@\@currname.\@currext\endcsname{#1}}
\def\@providesfile#1[#2]{%
    \expandafter\xdef\csname ver@#1\endcsname{#2}%
  \endgroup}
\def\@latex@info#1{}
\def\@font@info#1{}
\def\ClassInfo#1#2{}
\def\PackageInfo#1#2{}
\RequirePackage[log-declarations=false]{xparse}
\RequirePackage{ctexhook}

\ExplSyntaxOn
\cs_new:Npn \__fonttest_close_msg:nn #1#2
  { \msg_redirect_name:nnn {#1} {#2} { none } }
\__fonttest_close_msg:nn { LaTeX / xparse } { not-single-char     }
% \__fonttest_close_msg:nn { fontspec       } { defining-font       }
% \__fonttest_close_msg:nn { fontspec       } { no-scripts          }
\__fonttest_close_msg:nn { unicode-math   } { patch-macro         }
\ctex_at_end_package:nn { geometry } { \def\Gm@showparams#1{} }
\ExplSyntaxOff


\documentclass{article}
\usepackage{amsmath,physics,unicode-math,xeCJK,slashed,geometry,xcolor,hyperref}

% \geometry{paperwidth=30cm, paperheight=80cm, margin=2.54cm}
\geometry{a3paper, margin=2.54cm}
\unimathsetup{math-style=ISO, bold-style=ISO, mathrm=sym}
\hypersetup{colorlinks=true, urlcolor=MaterialIndigo, citecolor=MaterialTeal}
\definecolor{MaterialIndigo}{HTML}{3F51B5}
\definecolor{MaterialTeal}{HTML}{009688}

\setmainfont{FiraGO}[BoldFont=* SemiBold]
\setmonofont{Iosevka}
\setmathfont{FiraMath-Regular.otf}[Path=../release/fonts/, BoldFont=*]
\setCJKmainfont{Source Han Sans SC}

% \setmainfont{XITS}
% \setmonofont{TeX Gyre Cursor}
% \setmathfont{XITS Math}
% \setmathfont{STIX Two Math}

\def\bm#1{\symbfit{#1}}

\def\ee{\symrm{e}}
\def\ii{\symrm{i}}
\def\pp{\symrm{\pi}}

\def\dD{\symrm{D}}

\def\BesselJ{\symrm{J}}
\def\EulerGamma{\symrm{\Gamma}}

% TODO: script letters are not available
\def\lagL{\symscr{L}}
\def\actS{\symscr{S}}
\def\maniM{\symscr{M}}
\def\info{\symscr{o}}

\def\realR{\symbb{R}}

\def\pd{\partial}
\def\Box{\mdlgwhtsquare}
\def\hodge{\ast} % TODO: \star is not available
\def\trans{{\symrm{T}}}

\def\infint{\int_{-\infty}^{+\infty}}
\def\fhalf{\frac{1}{2}}

\title{\textbf{Formulas Test for Fira Math}}
\author{Xiangdong Zeng}

\begin{document}

\maketitle

\section{Ramanujan's formulas}

Ramanujan--Sato series \cite{wiki:ramanujan-sato-series}
\begin{align}
  \frac{1}{\pp}
  &= \frac{2\sqrt2}{99^2} \sum_{k=0}^{\infty} \frac{(4k)!}{k!^4} \frac{26390k+1103}{396^{4k}} \\
  &= \frac{12}{(640320)^{3/2}} \sum_{k=0}^{\infty}
     \frac{(6k)! (545140134k+13591409)}{(3k)! (k!)^3 (-2625 3741 2640 7680 00)^k} \\
  &= \frac{\sqrt{n}}{3} \sum_{m=0}^{\infty} \qty\big[6v(k)m + \bar{G}(k_1,\,k)] \, b_m c^m (k).
\end{align}
Rogers--Ramanujan continued fraction \cite{mathworld:rogers-ramanujan-continued-fraction}
\begin{equation}
  r = \cfrac{1}{1+\cfrac{\ee^{-2\pp}}{1+\cfrac{\ee^{-4\pp}}{1+\cdots}}}
    = \frac{\fourthroot{5}\sqrt{\phi}-\phi}{\ee^{-2\pp/5}} \approx 1.
\end{equation}
Ramanujan's Integral I \cite{mathworld:ramanujan-integral}
\begin{equation}
  \infint \frac{\BesselJ_{\mu+\xi}(x)}{x^{\mu+\xi}} \frac{\BesselJ_{\nu-\xi}(y)}{y^{\nu-\xi}}
          \ee^{\ii t\xi} \dd{\xi}
  = \qty[\frac{2\cos(\fhalf t)}{x^2\ee^{-\ii t/2}+y^2\ee^{\ii t/2}}]^{(\mu+\nu)/2}
    \cdot \BesselJ_{\mu+\nu} \sqrt{2\cos(\fhalf t)\qty(x^2\ee^{-\ii t/2}+y^2\ee^{\ii t/2})}
    \cdot \ee^{\ii t(\nu-\mu)/2}.
\end{equation}
Ramanujan's Integral II
\begin{align}
     \int_0^\infty \frac{\dd{x}}{(1+x^2)\,(1+r^2x^2)\,(1+r^4x^2)\cdots}
  &= \frac{\pp}{2\,(1+r+r^3+r^6+r^{10}+\cdots)}; \\[1ex]
     \int_0^\infty \frac{(1+arx)\,(1+ar^2x)\cdots}{(1+x^2)\,(1+r^2x^2)\,(1+r^4x^2)\cdots}
     x^{n-1} \dd{x}
  &= \frac{\pp}{\sin n\pp} \prod_{m=1}^\infty \frac{(1-r^{m-n})\,(1-ar^m)}{(1-r^m)\,(1-ar^{m-n})}.
\end{align}
Ramanujan log-trigonometric integrals \cite{mathworld:ramanujan-log-trigonometric-integrals}
\begin{equation}
  \int_0^{\frac{\pp}{2}}
    \frac{\sqrt{\fhalf \sqrt{\frac{\ln^2\cos\theta}{\theta^2+\ln^2\cos\theta}} + \fhalf}}%
         {\fourthroot{\theta^2 + \ln^2\cos\theta}} \dd{\theta}
  = \frac{\pp}{2\sqrt{\ln 2}}.
\end{equation}
Generalization:
\begin{align}
  R_n^- &= \frac{2}{\pp} \int_0^{\pp/2} \qty(\theta^2+\ln^2\cos\theta)^{-2^{-n-1}}
           \sqrt{\fhalf+\fhalf\sqrt{\fhalf+\cdots+\fhalf\sqrt{
                                    \frac{\ln^2\cos\theta}{\theta^2+\ln^2\cos\theta}}}} \dd{\theta}
         = (\ln 2)^{-2^{-n}}, \\[1.2ex]
  R_n^+ &= \frac{2}{\pp} \int_0^{\pp/2} \qty(\theta^2+\ln^2\cos\theta)^{2^{-n-1}}
           \sqrt{\fhalf+\fhalf\sqrt{\fhalf+\cdots+\fhalf\sqrt{
                                    \frac{\ln^2\cos\theta}{\theta^2+\ln^2\cos\theta}}}} \dd{\theta}
         = (\ln 2)^{2^{-n}}.
\end{align}
Ramanujan cos/cosh identity \cite{mathworld:ramanujan-cos-cosh-identity}
\begin{equation}
    \qty(1+2\sum_{n=1}^\infty \frac{\cos  n\theta}{\cosh n\pp})^{-2}
  + \qty(1+2\sum_{n=1}^\infty \frac{\cosh n\theta}{\cosh n\pp})^{-2}
  = \frac{2\EulerGamma^4\qty(\frac{3}{4})}{\pp}.
\end{equation}
See also \cite{zhihu:ramanujan}.

\section{Lagrangian and action}

Electromagnetism
\begin{equation}
  \lagL(x) = j^\mu(x) A_\mu(x) - \frac{1}{4\mu_0} F_{\mu\nu}(x) F^{\mu\nu}(x).
\end{equation}
Electromagnetism using differential forms
\begin{equation}
  \actS[\bm{A}] = -\int_{\maniM} \qty(\fhalf\bm{F}\wedge\hodge\bm{F} + \bm{A}\wedge\hodge\bm{J}).
\end{equation}
Dirac Lagrangian
\begin{equation}
  \lagL = \ii\hbar c\bar{\psi}\slashed{\pd}\psi - mc^2\bar{\psi}\psi.
\end{equation}
Quantum electrodynamic (QED) Lagrangian
\begin{equation}
  \lagL_{\mathrm{QED}}
  = \ii\hbar c\bar{\psi}\slashed{\dD}\psi - mc^2\bar{\psi}\psi
  - \frac{1}{4\mu_0} F_{\mu\nu} F^{\mu\nu}.
\end{equation}
Quantum chromodynamic (QCD) Lagrangian
\begin{equation}
  \lagL_{\mathrm{QCD}}
  = \sum_n \qty(\ii\hbar c\bar{\psi}_n\slashed{\dD}\psi_n - mc^2\bar{\psi}_n\psi_n)
  - \frac{1}{4} G^\alpha{}_{\mu\nu} G_\alpha{}^{\mu\nu}.
\end{equation}
Standard Model Lagrangian
\begingroup
\def\L{{\symrm{L}}}
\def\R{{\symrm{R}}}
\def\hc{\mathrm{h.c.}}
\begin{align}
  \lagL_{\mathrm{SM}}
  = &{} - \frac{1}{4} B_{\mu\nu}B^{\mu\nu} - \frac{1}{8} \tr(\bm{W}_{\mu\nu}\bm{W}^{\mu\nu})
        - \frac{1}{2} \tr(\bm{G}_{\mu\nu}\bm{G}^{\mu\nu})
        \tag*{[$U(1)$, $SU(2)$ and $SU(3)$ gauge]} \\
    &{} + \qty(\bar{\nu}_\L,\,\bar{e}_\L) \, \tilde{\sigma}^\mu \ii \dD_\mu \mqty(\nu_\L \\ e_\L)
        + \bar{e}_\R \sigma^\mu \ii \dD_\mu e_\R + \bar{\nu}_\R \sigma^\mu \ii \dD_\mu \nu_\R + \hc
        \tag*{[lepton dynamical]} \\
    &{} - \frac{\sqrt{2}}{v} \,
          \underbrace{\qty[  \qty(\bar{\nu}_\L,\,\bar{e}_\L) \, \phi M^e e_\R
                           + \bar{e}_\R \bar{M}^e \bar{\phi} \, \mqty(\nu_\L \\ e_\L)]}%
            _{\text{electron, muon, tauon}}
     {} - \frac{\sqrt{2}}{v} \,
          \overbrace{\qty[  \qty(-\bar{e}_\L,\,\bar{\nu}_\L) \, \phi^* M^\nu \nu_\R
                          + \bar{\nu}_\R \bar{M}^\nu \phi^\trans \mqty(-e_\L \\ \nu_\L)]}%
            ^{\text{neutrino}}
        \tag*{[lepton mass]} \\
    &{} + \qty(\bar{u}_\L,\,\bar{d}_\L) \, \tilde{\sigma}^\mu \ii \dD_\mu \mqty(u_\L \\ d_\L)
        + \bar{u}_\R \sigma^\mu \ii \dD_\mu u_\R + \bar{d}_\R \sigma^\mu \ii \dD_\mu d_\R + \hc
        \tag*{[quark dynamical]} \\
    &{} - \frac{\sqrt{2}}{v} \,
          \underbrace{\qty[  \qty(\bar{u}_\L,\,\bar{d}_\L) \, \phi M^d d_\R
                           + \bar{d}_\R \bar{M}^d \bar{\phi} \, \mqty(u_\L \\ d_\L)]}%
            _{\text{down, strange, bottom}}
     {} - \frac{\sqrt{2}}{v} \,
          \overbrace{\qty[  \qty(-\bar{d}_\L,\,\bar{u}_\L) \, \phi^* M^u u_\R
                          + \bar{u}_\R \bar{M}^u \phi^\trans \mqty(-d_\L \\ u_\L)]}%
            ^{\text{up, charmed, top}}
        \tag*{[quark mass]} \\
    &{} + \overline{\dD_\mu\phi} \, \dD^\mu\phi
        - \frac{m_H^2}{2v^2} \qty(\bar{\phi}\phi-\frac{v^2}{2})^2
        \tag*{[Higgs dynamical and mass]}.
\end{align}
\endgroup
See \cite{wiki:lagrangian} and \cite{shifflett:lagrangian}.

\section{Maxwell's equations}

Differential equations (vaccum):
\begin{equation}
  \nabla\cdot\bm{E} = \frac{\rho}{\varepsilon_p} \qc
  \nabla\cdot\bm{B} = 0 \qc
  \nabla\times\bm{E} = -\pdv{\bm{B}}{t} \qc
  \nabla\times\bm{B} = \mu_0\bm{J} + \mu_0\varepsilon_0\pdv{\bm{E}}{t}.
\end{equation}
Integral equations (vaccum):
\begin{equation}
  \oiint_{\pd\Omega} \bm{E}\cdot\dd{\bm{S}}
    = \frac{1}{\varepsilon_0} \iiint_\Omega \rho\dd{V} \qc
  \oiint_{\pd\Omega} \bm{B}\cdot\dd{\bm{S}} = 0 \qc
  \oint_{\pd\Sigma} \bm{E}\cdot\dd{\bm{l}}
    = - \dv{t} \iint_\Sigma \bm{B}\cdot\dd{\bm{S}} \qc
  \oint_{\pd\Sigma} \bm{B}\cdot\dd{\bm{l}}
    = \iint_\Sigma \mu_0\bm{J}\cdot\dd{\bm{S}}
    + \dv{t}\iint_\Sigma \mu_0\varepsilon_0\bm{E}\cdot\dd{\bm{S}}.
\end{equation}
Differential equations (matter):
\begin{equation}
  \nabla\cdot\bm{D} = \rho_{\mathrm{f}} \qc
  \nabla\cdot\bm{B} = 0 \qc
  \nabla\times\bm{E} = -\pdv{\bm{B}}{t} \qc
  \nabla\times\bm{B} = \bm{J}_{\mathrm{f}} + \pdv{\bm{D}}{t}.
\end{equation}
Integral equations (matter):
\begin{equation}
  \oiint_{\pd\Omega} \bm{D}\cdot\dd{\bm{S}}
    = \iiint_\Omega \rho_{\mathrm{f}}\dd{V} \qc
  \oiint_{\pd\Omega} \bm{B}\cdot\dd{\bm{S}} = 0 \qc
  \oint_{\pd\Sigma} \bm{E}\cdot\dd{\bm{l}}
    = - \dv{t} \iint_\Sigma \bm{B}\cdot\dd{\bm{S}} \qc
  \oint_{\pd\Sigma} \bm{H}\cdot\dd{\bm{l}}
    = \iint_\Sigma \bm{J}_{\mathrm{f}}\cdot\dd{\bm{S}}
    + \dv{t}\iint_\Sigma \bm{D}\cdot\dd{\bm{S}}.
\end{equation}
Tensor calculus (Lorenz gauge):
\begin{gather}
  B_{ij} = \pd_{[i} A_{j]} = \nabla_{[i} A_{j]} \qc
  E_i = - \pdv{A_i}{t} - \pd_i\phi = -\pdv{A_i}{t} - \nabla_i\phi, \\
  - \frac{1}{\sqrt{h}} \pd_i\sqrt{h} \qty(\pd^i\phi + \pdv{A^i}{t})
    = - \nabla_i\nabla^i\phi - \pdv{t}\nabla_i A^i = \frac{\rho}{\varepsilon_0}, \\
  - \frac{1}{\sqrt{h}} \pd_i \qty(\sqrt{h}h^{im}h^{jn}\pd_{[m}A_{n]})
  + \frac{1}{c^2} \pdv{t} \qty(\pdv{A^j}{t} + \pd^j\phi)
    = - \nabla_i\nabla^i A^j + \frac{1}{c^2}\pdv[2]{A^j}{t} + R^j_i A^i
      + \nabla^j \qty(\nabla_i A^i + \frac{1}{c^2}\pdv{\phi}{t}) = \mu_0 J^i.
\end{gather}
Differential forms (Lorenz gauge):
\begin{equation}
  F = \dd{A} \qc
  \dd{\hodge A} = 0 \qc
  \hodge\Box A = \mu_0 J.
\end{equation}
See \cite{wiki:maxwell-equations}.

\section{Frenet--Serret frame}

Definition
\begin{equation}
  \bm{T}(s) = \dot{\bm{r}}(s) \qc
  \bm{N}(s) = \frac{\ddot{\bm{r}}(s)}{\norm{\ddot{\bm{r}}(s)}_{\realR^3}} \qc
  \bm{B}(s) = \bm{T}(s)\vectimes\bm{N}(s)
            = \frac{\dot{\bm{r}}(s)\vectimes\ddot{\bm{r}}(s)}{\norm{\ddot{\bm{r}}(s)}_{\realR^3}}.
\end{equation}
Frenet--Serret formulas
\begin{equation}
  \mqty[\dot{\bm{T}}(s) \\[0.4ex] \dot{\bm{N}}(s) \\[0.4ex] \dot{\bm{B}}(s)]
  = \mqty[0 & \kappa & 0 \\[0.5ex] -\kappa & 0 & \tau \\[0.5ex] 0 & -\tau & 0] \,
    \mqty[\bm{T}(s) \\[0.5ex] \bm{N}(s) \\[0.5ex] \bm{B}(s)] \qc \text{where curvature\ }
  \kappa = \norm{\ddot{\bm{r}}(s)}_{\realR^3} \qc \text{torsion\ }
  \tau   = \frac{\det\qty\big(\dot{\bm{r}}(s),\,\ddot{\bm{r}}(s),\,\dddot{\bm{r}}(s))}%
                {\norm{\ddot{\bm{r}}(s)}_{\realR^3}^2}
         = \frac{\qty\big(\dot{\bm{r}}(s)\vectimes\ddot{\bm{r}}(s))\cdotp\dddot{\bm{r}}(s)}%
                {\norm{\ddot{\bm{r}}(s)}_{\realR^3}^2}.
\end{equation}
Taylor's expansion
\begin{equation}
  \bm{r}(s) = \bm{r}(0) + \qty[s-\frac16 s^3\kappa^2(0)] \, \bm{T}(0)
    + \qty[\frac12 s^2\kappa(0) + \frac16 s^3\kappa'(0)] \, \bm{N}(0)
    + \qty[\frac16 s^3\kappa(0)\tau(0)] \, \bm{B}(0) + \info\qty(s^3).
\end{equation}
Generalization
\begin{equation}
  \mqty(\dot{\bm{e}}_1(s) \\[0.4ex] \vdots \\[0.4ex] \dot{\bm{e}}_n(s))
  = \mqty(0 & \chi_1(s) & & 0 \\[0.5ex] -\chi_1(s) & \ddots & \ddots & \\[0.5ex]
            & \ddots & 0 & \chi_{n-1}(s) \\[0.5ex] 0 & & -\chi_{n-1}(s) & 0) \,
    \mqty(\bm{e}_1(s) \\[0.5ex] \vdots \\[0.5ex] \bm{e}_n(s)) \qc
  \text{where generalized curvature\ }
  \chi_i(s) = \frac{\ev{\dot{\bm{e}}_i(s),\,\bm{e}_{i+1}(s)}}%
                   {\norm{\dot{\bm{r}}(s)}_{\realR^n}}.
\end{equation}
See \cite{wiki:frenet-serret-formulas} and \cite{xie:tensor}.

\begin{thebibliography}{99}
  \bibitem{mathworld:ramanujan-cos-cosh-identity}
    \newblock Eric W. Weisstein,
    \newblock \href{http://mathworld.wolfram.com/RamanujanCosCoshIdentity.html}%
                   {\textbf{Ramanujan Cos/Cosh Identity}}.
    \newblock \textit{MathWorld--A Wolfram Web Resource}.

  \bibitem{mathworld:ramanujan-integral}
    \newblock Eric W. Weisstein,
    \newblock \href{http://mathworld.wolfram.com/RamanujansIntegral.html}%
                   {\textbf{Ramanujan's Integral}}.
    \newblock \textit{MathWorld--A Wolfram Web Resource}.

  \bibitem{mathworld:ramanujan-log-trigonometric-integrals}
    \newblock Eric W. Weisstein,
    \newblock \href{http://mathworld.wolfram.com/RamanujanLog-TrigonometricIntegrals.html}%
                   {\textbf{Ramanujan Log-Trigonometric Integrals}}.
    \newblock \textit{MathWorld--A Wolfram Web Resource}.

  \bibitem{mathworld:rogers-ramanujan-continued-fraction}
    \newblock Tito Piezas III and Eric W. Weisstein,
    \newblock \href{http://mathworld.wolfram.com/Rogers-RamanujanContinuedFraction.html}%
                   {\textbf{Rogers--Ramanujan Continued Fraction}}.
    \newblock \textit{MathWorld--A Wolfram Web Resource}.

  \bibitem{wiki:frenet-serret-formulas}
    \newblock Wikipedia contributors,
    \newblock \href{https://en.wikipedia.org/wiki/Frenet%E2%80%93Serret_formulas}%
                   {\textbf{Frenet--Serret formulas}}.
    \newblock \textit{Wikipedia, the free encyclopedia}.

  \bibitem{wiki:lagrangian}
    \newblock Wikipedia contributors,
    \newblock \href{https://en.wikipedia.org/wiki/Lagrangian_(field_theory)}%
                   {\textbf{Lagrangian (field theory)}}.
    \newblock \textit{Wikipedia, the free encyclopedia}.

  \bibitem{wiki:maxwell-equations}
    \newblock Wikipedia contributors,
    \newblock \href{https://en.wikipedia.org/wiki/Maxwell%27s_equations}%
                   {\textbf{Maxwell's equations}}.
    \newblock \textit{Wikipedia, the free encyclopedia}.

  \bibitem{wiki:ramanujan-sato-series}
    \newblock Wikipedia contributors,
    \newblock \href{https://en.wikipedia.org/wiki/Ramanujan%E2%80%93Sato_series}%
                   {\textbf{Ramanujan--Sato series}}.
    \newblock \textit{Wikipedia, the free encyclopedia}.

  \bibitem{shifflett:lagrangian}
    \newblock Jim Shifflett,
    \newblock \href{http://einstein-schrodinger.com/Standard_Model.pdf}%
                   {\textbf{Standard Model Lagrangian (including neutrino mass terms)}}.

  \bibitem{zhihu:ramanujan}
    \newblock 酱紫君,
    \newblock \href{https://www.zhihu.com/question/26292855/answer/266846441}%
                   {\textbf{有哪些美丽或神奇的理科公式?- 酱紫君的回答}}.
    \newblock 知乎.

  \bibitem{xie:tensor}
    \newblock 谢锡麟,
    \newblock \textbf{现代张量分析及其在连续介质力学中的应用}.
    \newblock 上海: 复旦大学出版社.
    \newblock ISBN: 978-7-309-10980-1.
\end{thebibliography}

\end{document}
