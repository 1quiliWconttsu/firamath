\documentclass[aspectratio=169]{beamer}
\usepackage{amsmath,unicode-math,physics,tensor,xeCJK,bookmark}
\useoutertheme{metropolis}
\useinnertheme{metropolis}
\usecolortheme{metropolis}
\usefonttheme{professionalfonts}

\setbeamerfont{title}{size=\Large, series=\bfseries}
\setbeamerfont{author}{size=\small}
\setbeamerfont{date}{size=\small}
\setbeamertemplate{footline}{\vspace*{0.3cm}}

\makeatletter
% https://tex.stackexchange.com/q/66519
\apptocmd{\beamer@@frametitle}{\only<1>{\bookmark[page=\the\c@page,level=3]{#1}}}{}{}

\unimathsetup{math-style=ISO, bold-style=ISO, mathrm=sym}

\setsansfont{FiraGO}[BoldFont=* SemiBold, Numbers=Monospaced]
%%<DEBUG>
\setmathfont{FiraMath-Regular.otf}[Path=../release/fonts/]
%%<RELEASE>
%%\setmathfont{FiraMath-Regular.otf}
%%<END>

\newCJKfontfamily\fontzhhans{Source Han Sans SC}
\newCJKfontfamily\fontzhhant{Source Han Sans TC}
\newCJKfontfamily\fontja{Source Han Sans}

\ExplSyntaxOn
\seq_const_from_clist:Nn \c_@@_weight_seq
  { Thin, UltraLight, ExtraLight, Light, Book, Regular, Medium, SemiBold, Bold, ExtraBold, Heavy, Ultra }
\seq_map_inline:Nn \c_@@_weight_seq
%%<DEBUG>
  { \setmathfont { FiraMath-#1.otf } [ Path = ../release/fonts/, version = #1 ] }
%%<RELEASE>
%%  { \setmathfont { FiraMath-##1.otf } [ version = ##1 ] }
%%<END>
\cs_new:Npn \MultipleWeights #1
  {
    \seq_map_inline:Nn \c_@@_weight_seq
      { \group_begin: \mathversion {##1} #1 \group_end: }
  }
\ExplSyntaxOff
\makeatother

\def\ii{\symrm{i}}
\def\pp{\symrm{\pi}}

\title{Fira Math}
\subtitle{Sans-serif font with Unicode math support}
\author{Xiangdong Zeng}
\date{2020/01/13\quad v0.3.3}

\begin{document}

\maketitle

\begin{frame}{Basic examples (I)}
\begin{itemize}
  \item Covariant derivative:
    \[
      \nabla \symbf{X} = \tensor{X}{^\alpha_{;\beta}} \pdv{x^\alpha} \otimes \dd{x^\beta}
                       = \qty(\tensor{X}{^\alpha_{,\beta}} + \Gamma^{\alpha}_{\beta\gamma} \, X^\gamma) \,
                         \pdv{x^\alpha} \otimes \dd{x^\beta}
    \]
  \item Einstein's field equations:
    \[ G_{\mu\nu} \equiv R_{\mu\nu} - \frac{1}{2} R g_{\mu\nu} = \frac{8\pi G}{c^4} T_{\mu\nu} \]
  \item Schwarzschild metric:
    \[
      c^2 \dd{\tau}^2 = \qty(1-\frac{r_{\mathrm{s}}}{r}) \, c^2 \dd{t}^2
                      - \qty(1-\frac{r_{\mathrm{s}}}{r})^{-1} \dd{r}^2
                      - r^2 \underbrace{\qty(\dd{\theta}^2 + \sin^2 \theta \dd{\varphi}^2)}_{\dd{\Omega}^2}
    \]
  \item Einstein--Hilbert action:
    \[ S = \frac{1}{2\kappa} \int R \sqrt{-g} \dd[4]{x} \]
\end{itemize}
\end{frame}

\begin{frame}{Basic examples (II)}
\begin{itemize}
  \item Case $n=1$
    \small
    \[
      \int_0^{\frac{\pp}{2}}
        \frac{\sqrt{\frac12 \sqrt{\frac{\ln^2\cos\theta}{\theta^2+\ln^2\cos\theta}} + \frac12}}%
            {\fourthroot{\theta^2 + \ln^2\cos\theta}} \dd{\theta}
      = \frac{\pp}{2\sqrt{\ln 2}}
    \]
  \item Generalization:
    \small\vspace{1ex}
    \[
      \begin{cases}
        \smash[t]{\displaystyle
          R_n^- = \frac{2}{\pp} \int_0^{\pp/2} \qty(\theta^2+\ln^2\cos\theta)^{-2^{-n-1}}
                  \sqrt{\frac12+\frac12\sqrt{\frac12+\cdots+\frac12\sqrt{
                        \frac{\ln^2\cos\theta}{\theta^2+\ln^2\cos\theta}}}} \dd{\theta}
                = (\ln 2)^{-2^{-n}}} \\[3ex]
        \smash[b]{\displaystyle
          R_n^+ = \frac{2}{\pp} \int_0^{\pp/2} \qty(\theta^2+\ln^2\cos\theta)^{2^{-n-1}}
                  \sqrt{\frac12+\frac12\sqrt{\frac12+\cdots+\frac12\sqrt{
                        \frac{\ln^2\cos\theta}{\theta^2+\ln^2\cos\theta}}}} \dd{\theta}
                = (\ln 2)^{2^{-n}}}
      \end{cases}
    \]
\end{itemize}
\end{frame}

\begin{frame}{Using with CJK fonts}
\begin{itemize}
  \item {\fontzhhans 【留数定理】全纯函数 $f$ 在若尔当曲线 $\gamma$ 上的积分为:}
    \[
      \oint_\gamma f(z) \dd{z}
      = 2\pp\ii \sum_{k=1}^n \Res_{z=a_k} f(z).
    \]
  \item {\fontzhhant 【留數定理】全純函數 $f$ 在若爾當曲線 $\gamma$ 上的積分為:}
    \[
      \oint_\gamma f(z) \dd{z}
      = 2\pp\ii \sum_{k=1}^n \Res_{z=a_k} f(z).
    \]
  \item {\fontja 【留数定理】ジョルダン曲線 $\gamma$ に沿う正則関数 $f$ の積分は、}
    \[
      \oint_\gamma f(z) \dd{z}
      = 2\pp\ii \sum_{k=1}^n \Res_{z=a_k} f(z).
    \]
\end{itemize}
\end{frame}

\begin{frame}{Multiple weights (preview)}
\centering
\everymath{\displaystyle}
\begin{tabular}{p{5cm}l}
  \mathversion{Thin}       $ \pdv{\alpha} \sin\alpha =  \cos  \alpha       $ & \mathversion{Medium}     $ \int \sin x \dd{x} = -\cos x                   + C_1 $ \\[12pt]
  \mathversion{UltraLight} $ \pdv{\beta } \cos\beta  = -\sin  \beta        $ & \mathversion{SemiBold}   $ \int \cos y \dd{y} =  \sin y                   + C_2 $ \\[12pt]
  \mathversion{ExtraLight} $ \pdv{\gamma} \tan\gamma =  \sec^2\gamma       $ & \mathversion{Bold}       $ \int \tan z \dd{z} = -\ln|\cos z \, |          + C_3 $ \\[12pt]
  \mathversion{Light}      $ \pdv{\theta} \cot\theta = -\csc^2\theta       $ & \mathversion{ExtraBold}  $ \int \cot p \dd{p} =  \ln|\sin p \, |          + C_4 $ \\[12pt]
  \mathversion{Book}       $ \pdv{\phi  } \sec\phi   =  \tan\phi \sec\phi  $ & \mathversion{Heavy}      $ \int \sec q \dd{q} =  \ln|\sec q + \tan q \, | + C_5 $ \\[12pt]
  \mathversion{Regular}    $ \pdv{\zeta } \csc\zeta  = -\cot\zeta\csc\zeta $ & \mathversion{Ultra}      $ \int \csc r \dd{r} = -\ln|\csc r + \cot r \, | + C_6 $
\end{tabular}
\end{frame}

\end{document}
